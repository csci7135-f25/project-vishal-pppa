%% This is an abbreviated template from http://www.sigplan.org/Resources/Author/.

\documentclass[acmsmall,review,nonacm]{acmart}
\begin{document}

%%
%% The "title" command has an optional parameter,
%% allowing the author to define a "short title" to be used in page headers.
\title{The Name of the Title is Hope}

%%
%% The "author" command and its associated commands are used to define
%% the authors and their affiliations.
%% Of note is the shared affiliation of the first two authors, and the
%% "authornote" and "authornotemark" commands
%% used to denote shared contribution to the research.
\author{Vishal Vunnam}
\email{vivu9928@colorado.edu}
\author{Vishal Vunnam}
\email{vivu9928@colorado.edu}
\affiliation{%
  \institution{University of Colorado Boulder}
  \country{USA}
}


%%
%% The abstract is a short summary of the work to be presented in the
%% article.
\begin{abstract}
Static Program Analysis is a cornerstone technique in software engineering for ensuring code reliability and security. Traditional static analysis, using abstract interpretation, represents the program as a set of mathematical constraints over abstract domains. These analyzers are sound, but often imprecise due to the undecidability of program termination, leading to excessive false positives.
Recent work has explored using Language Models to improve the precision of widening-based static analyzers by predicting loop invariants and abstraction heuristics. However, these approaches explode in both context size and computation time as the number of program variables increases, making them infeasible for real-world programs.
In this work, we propose SLICE-ABSINT, a novel approach to static analysis that leverages Language Models to guide the analysis process while maintaining scalability. By slicing the program based on the dependencies of diverging abstract states, we dynamically prune the context window to strictly relevant code paths before querying the neural oracle. This semantic filtering prevents "context pollution," ensuring the model focuses solely on variables affecting the widening decision.
Our evaluation on the SV-COMP benchmark suite demonstrates that SLICE-ABSINT reduces token consumption by XXX while improving invariant prediction accuracy by XXX compared to full-context baselines, establishing that context reduction is a prerequisite for scalable neurosymbolic verification.
\end{abstract}
%%
%% This command processes the author and affiliation and title
%% information and builds the first part of the formatted document.
\maketitle

\section{Introduction}
While recent works like LLMxCPG leverage slicing to enhance LLM-based vulnerability detection, they lack formal guarantees. SLICE-ABSINT is the first framework to apply semantic slicing specifically to the Abstract Widening Operator. Unlike heuristic bug-finders, our slices are formally proven to preserve the observational equivalence of the abstract domain, ensuring that the reduced context never compromises the soundness of the final verification."

\section{Overview}

\section{[Contribution 1]}

\section{[Contribution 2]}

\section{Evaluation}

\section{Related Work}

\section{Conclusion}

%%
%% The acknowledgments section is defined using the "acks" environment
%% (and NOT an unnumbered section). This ensures the proper
%% identification of the section in the article metadata, and the
%% consistent spelling of the heading.
\begin{acks}
TBD
\end{acks}

%%
%% The next two lines define the bibliography style to be used, and
%% the bibliography file.
\bibliographystyle{ACM-Reference-Format}
\bibliography{paper}
\end{document}
